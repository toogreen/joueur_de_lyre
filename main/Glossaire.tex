% ATTENTION - LIRE - CETTE PAGE PROVIENT DU DOCUMENT SOURCE DE DIDIER ROCHE ET N'EST PAS UTILISEE PRESENTEMENT POUR LA MARIRIOU. 
% JE L'AI LAISSEE ICI JUSTE EN CAS DE FUTURE AJOUTS ET/OU MODIFICATIONS SI NECESSAIRE.

\entreeglossaire{Antivirus}{Un antivirus -- anti-virus -- est un logiciel censé protéger un micro-ordinateur contre les programmes néfastes appelés virus, vers, troyens, macrovirus, etc.}
\entreeglossaire{Avahi}{Avahi est une implémentation de ZéroConf permettant la découverte de services offerts par le réseau comme les imprimantes disponibles, partager de la musique et l'écouter à distance, parler par messagerie Internet aux autres ordinateurs hôtes de votre réseau, et tout cela sans configuration préalable.}
\entreeglossaire{BIOS}{Le Basic Input Output System -- système élémentaire d'entrée/sortie -- est un ensemble de fonctions, contenues dans la mémoire morte -- ROM -- de la carte mère servant à faire des opérations basiques. Celui-ci émet notamment les premières commandes au système durant la phase de démarrage, pour indiquer par exemple sur quel disque et à quel endroit se trouve le MBR.}
\entreeglossaire{Boot loader}{Terme anglais de «~Chargeur d'amorçage~». Se référer à la définition de ce dernier.}
\entreeglossaire{Chargeur d'amorçage}{Également appelé «~Boot loader~», ce programme permet de choisir quel système d'exploitation doit être lancé. Dans le cas de Windows, il s'agit du programme NTLDR, dans le cas d'un système en multiboot -- possibilité de démarrer plusieurs systèmes d'exploitation sur un même ordinateur -- Lilo dans les cas simples -- Windows et Linux -- ou GRUB dans les cas plus sophistiqués --- tous systèmes supportés.}
\entreeglossaire{Codec}{Le mot-valise «~Codec~» est construit d'après les mots «~codeur~» et «~décodeur~». Il s'agit d'un procédé permettant de compresser et/ou de décompresser un signal audio ou vidéo, le plus souvent en temps réel. Les codecs peuvent être partagés entre plusieurs logiciels de lecture audio ou vidéo.}
\entreeglossaire{Console}{Le mot console est un synonyme de «~terminal~». Se référer à la définition de ce dernier.}
\entreeglossaire{Défragmentation}{La défragmentation réorganise physiquement le contenu du disque pour remettre ensemble et dans l'ordre les fragments ou morceaux éparpillés d'un fichier. Ce processus d'élimination essaye également de créer une grande région d'espace libre pour retarder à nouveau la fragmentation du système de fichiers.}
\entreeglossaire{Démon}{Le terme démon -- ou daemon en anglais -- désigne un type de programme. Le terme a été créé par les inventeurs d'Unix pour se référer à un processus qui s'exécute en arrière-plan plutôt que sous le contrôle direct d'un utilisateur. Les démons sont souvent démarrés lors du chargement du système d'exploitation, et servent en général à répondre à des requêtes du réseau, à l'activité du matériel ou à d'autres programmes en exécutant certaines tâches.}
\entreeglossaire{DHCP}{Le Dynamic Host Configuration Protocol est un terme anglais désignant un protocole réseau dont le rôle est d'assurer la configuration automatique des paramètres TCP/IP d'une station, notamment en lui assignant automatiquement une adresse IP, une passerelle et un masque de sous-réseau.}
\entreeglossaire{DNS}{Le Domain Name System -- ou système de noms de domaine -- est un système permettant d'établir une correspondance entre une adresse IP -- exemple : 80.82.17.133 -- et un nom de domaine -- exemple : \url{http://www.framabook.org} -- et, plus généralement, de trouver une information à partir d'un nom de domaine.}
\entreeglossaire{Driver}{Terme anglais de «~pilote~». Se référer à sa définition.}
\entreeglossaire{Ethernet (câble)}{Le câble ethernet le plus connu est certainement le câble RJ11, équipant la plupart des téléphones. Cette interface physique est souvent utilisée pour terminer les câbles de type paire torsadée, et sa version RJ45 est celle choisie pour équiper la plupart des cartes réseaux actuelles d'ordinateur.}
\entreeglossaire{Firewall}{Terme anglais de «~pare-feu~». Se référer à la définition de ce dernier.}
\entreeglossaire{Geek}{Terme anglais se prononçant [gi:k], un geek est une personne passionnée, voire obsédée, par un domaine précis (la plupart du temps rapporté à l'informatique et aux nouvelles technologies). À l'origine, en anglais le terme signifiait «~fada~», soit une variation argotique de «~fou~».}
\entreeglossaire{Glisser-déposer}{Pour réaliser un glisser-déposer, il faut d'abord sélectionner un ou plusieurs éléments, puis maintenir appuyé, lors du déplacement, le bouton gauche de la souris et enfin le relâcher sur le point d'arrivée.}
\entreeglossaire{Identifiant}{Appelé également «~nom d'utilisateur~» et «~login~» en anglais.}
\entreeglossaire{L'Internet --- et non internet}{Un internet est un réseau, l'Internet -- ne pas oublier le «~l'~» et la majuscule -- est «~le réseau des réseaux~». L'adjectif, lui, est cependant bien «~Internet~», comme dans site Internet. Ce  réseau informatique à l'échelle du monde, reposant sur le protocole de communication IP, rend accessible au public des services comme le courrier électronique et le web. Vous pourrez reprendre un grand nombre de présentateurs télévisés --- qui a dit que ce n'est pas la première fois ? :-)}
\entreeglossaire{IP (adresse)}{Une adresse IP -- IP pour Internet Protocol -- est le numéro qui identifie chaque ordinateur sur l'Internet, et plus généralement, l'interface avec le réseau de tout matériel informatique -- routeur, imprimante -- connecté à un réseau informatique utilisant le protocole Internet.}
\entreeglossaire{Kernel}{Terme anglais de «~noyau~». Voir plus bas la définition de ce dernier.}
\entreeglossaire{Logiciels propriétaires --- notion opposée à celle des «~logiciels libres~»}{Ce sont des logiciels dont l'utilisation est limitée d'une manière très précise par un contrat de licence.}
\entreeglossaire{Masque de sous-réseau}{Un masque de sous-réseau permet d'identifier un sous-réseau. En l'appliquant sur l'adresse IP de la machine, il permet de déterminer si certaines machines appartiennent, ou non, au même réseau alors qu'elles sont connectées physiquement.}
\entreeglossaire{MBR}{Le Master Boot Record, également appelé «~Zone d'amorcage~», est le nom donné par le BIOS au premier secteur adressable d'un disque dur. Ce dernier peut contenir le «~Chargeur d'amorçage~» ou encore «~Boot loader~» ou une adresse pour l'atteindre.}
\entreeglossaire{Menu contextuel}{Ce menu s'obtient en cliquant avec le bouton droit de la souris sur un objet. Il contient un choix de fonctions variant selon l'objet et son contexte.}
\entreeglossaire{Noyau}{En informatique, un noyau de système d'exploitation abrégé en noyau -- kernel en anglais -- est la partie fondamentale de tous les systèmes d'exploitation. Elle gère les ressources de l'ordinateur et permet aux différents composants - matériels et logiciels - de communiquer entre eux.}
\entreeglossaire{Operating System}{Souvent appelé OS. Terme anglais de Système d'exploitation. Voir cette définition.}
\entreeglossaire{Pare-feu}{Un pare-feu est un élément du réseau informatique, logiciel et/ou matériel, qui a pour fonction de faire respecter la politique de sécurité du réseau, celle-ci définissant les types de communication autorisés ou interdits. Il a pour principale tâche de contrôler le trafic entre différentes zones de confiance, en filtrant les flux de données qui y transitent. Généralement, les zones de confiance incluent l'Internet -- une zone dont la confiance est nulle -- et au moins un réseau interne --- une zone dont la confiance est plus importante.}
\entreeglossaire{Passerelle}{En informatique, une passerelle -- en anglais, gateway -- est un dispositif permettant de relier deux réseaux informatiques différents, comme par exemple un réseau local et l'Internet. Ainsi, plusieurs ordinateurs ou l'ensemble du réseau local peuvent accéder à l'Internet par l'intermédiaire de la passerelle. Le plus souvent, elle sert aussi de pare-feu, ce qui permet de contrôler tous les transferts de données entre le local et l'extérieur.}
\entreeglossaire{Phishing}{Contraction des mots anglais «~PHreaking~», signifiant le détournement d'un système téléphonique, et de «~fISHING~» qui est la pêche à la ligne. Le phishing, appelé hameçonnage en français, correspond à une technique utilisée par des fraudeurs pour obtenir des renseignements personnels dans le but de perpétrer une usurpation d'identité. La technique consiste à faire croire à la victime qu'elle s'adresse à un tiers de confiance -- banque, administration, etc. -- afin de lui soutirer des renseignements personnels : mot de passe, numéro de carte de crédit, date de naissance, etc.}
\entreeglossaire{Pilote}{Un pilote informatique souvent abrégé en pilote et quelquefois nommé driver est un programme, souvent accompagné de fichiers ASCII -- ou fichiers «~texte brut~» -- de configuration, destiné à permettre à un autre programme -- souvent un système d'exploitation -- d'interagir avec un périphérique. En général, chaque périphérique a son propre pilote. Sans pilote, l'imprimante ou la carte graphique ne pourraient fonctionner.}
\entreeglossaire{Port}{Il s'agit d'un port logiciel mettant en œuvre un service. Il existe divers numéros de ports : par exemple 25 pour le SMTP, 110 pour le POP, 80 pour le HTTP, etc. Ce terme peut également correspondre à une prise physique permettant de connecter un périphérique.}
\entreeglossaire{Protocole}{Dans les réseaux informatiques et les télécommunications, un protocole de communication est une spécification de plusieurs règles pour un type de communication particulier.}
\entreeglossaire{RSS} {Pouvant porter les noms de Rich Site Summary, Really Simple Syndication ou encore RDF Site Summary, RSS est un format de description et de publication pour les contenus des sites Internet.}
\entreeglossaire{Secteur d'amorce}{Zone particulière d'un disque dur ou d'une disquette permettant de démarrer le système d'exploitation d'un ordinateur. Le démarrage de l'ordinateur est appelé «~boot~» en anglais, d'où le terme «~secteur de boot~».}
\entreeglossaire{Services}{Terme utilisé sur Microsoft Windows. Équivalent de démon ; voir cette définition.}
\entreeglossaire{Système d'exploitation}{Désigné par ses initiales -- SE -- ou, plus communément encore, par le terme anglais OS. Ensemble de logiciels permettant d'utiliser un ordinateur et ses divers périphériques. Les systèmes d'exploitation les plus connus sont Windows, Mac OS et les distributions GNU/Linux.}
\entreeglossaire{Terminal}{Un terminal, ou «~text teletype~» -- TTY -- est une fenêtre de texte dans laquelle on peut entrer des instructions en mode texte. Ils présentent les «~sorties~» -- c'est à dire ses réponses -- uniquement sous forme textuelle et disposent simplement d'un clavier pour les «~entrées~» --- terme utilisé pour exprimer la manière dont laquelle vous pouvez communiquer avec lui. Un exemple de terminal texte qui fut répandu en France est le Minitel, lequel est relié aux serveurs par l'intermédiaire de la ligne téléphonique.}
\entreeglossaire{Texte brut}{Texte sans mise en forme c'est-à-dire, par exemple, sans gras, sans italique ou sans couleur.}
\entreeglossaire{Thème}{Habillages ou thèmes d'une application, appelés «~skins~» --- des peaux en anglais. Il s'agit de la définition de l'apparence graphique --- couleurs choisies, forme des boutons, icônes...}
\entreeglossaire{Wi-Fi}{Le Wi-Fi -- également orthographié WiFi, Wifi ou encore wifi -- est une technologie de réseau informatique sans fil mise en place pour fonctionner en réseau interne et, depuis, devenue un moyen d'accès à haut débit à l'Internet.}
\entreeglossaire{ZéroConf}{Zero Configuration Networking est le nom d'un ensemble de technologies permettant à plusieurs ordinateurs de communiquer sans configuration. Le but est d'obtenir un réseau IP fonctionnel sans dépendance d'une infrastructure -- serveur DHCP, serveur DNS, etc. -- ou d'une expertise réseau.}
\entreeglossaire{Zone d'amorçage}{Synonyme de «~MBR~». Se reporter, plus haut, à cette définition.}
\entreeglossaire{DRM}{Les Digital Rights Management, terme anglais de Gestion des Droits Numériques -- GDN -- a pour objectif de contrôler par des mesures techniques de protection, très restrictives et contraignantes, l'utilisation qui est faite des œuvres numériques.}
\entreeglossaire{Troll}{Le terme troll est utilisé pour désigner une personne, ou un groupe de personnes, participant à un espace de discussion -- de type forum par exemple -- cherchant à détourner insidieusement le sujet d'une discussion pour générer des conflits en incitant à la polémique et à la provocation.}
\entreeglossaire{Icône}{Petit pictogramme représentant une action, un objet, un logiciel, un type de fichier, etc. Les icônes ont dans un premier temps servi d'outils pour rendre les environnements graphiques informatiques plus simple d'utilisation aux novices.}
\entreeglossaire{GDN}{Terme français de DRM, largement plus utilisé. Voir la définition plus haut.}
\entreeglossaire{Drag'n'Drop}{Terme anglais de glisser-déposer, voir la définition de ce dernier.}
\entreeglossaire{Logiciels libres}{Un logiciel libre se dit d'un logiciel qui donne le droit à toute personne possédant une copie, de l'utiliser, de l'étudier, de le modifier et de le redistribuer. Ce droit est souvent donné par une Licence Libre. Tout ceci se réfère directement au mouvement du Libre.}
\entreeglossaire{RAM}{Random Access Memory : mémoire à accès aléatoire. Type de mémoire vive. Se référer à la définition de cette dernière.}
\entreeglossaire{ROM}{Read-Only Memory : mémoire à lecture seule. Type de mémoire morte. Se reporter à la définition de celui-ci.}
\entreeglossaire{Mémoire vive}{La mémoire vive, dite mémoire système ou encore mémoire volatile est la mémoire dans laquelle un ordinateur place les données lors de leur traitement. C'est donc cette mémoire d'un accès très rapide qui est utilisée lorsque l'ordinateur est allumé. Celle-ci est complètement vidée à l'extinction de l'ordinateur. Cette notion est à opposer à la mémoire morte.}
\entreeglossaire{Mémoire morte}{Une mémoire morte est une mémoire non volatile, c'est-à-dire une mémoire qui ne s'efface pas lorsque l'appareil qui la contient n'est plus alimenté en électricité. Le type le plus connu du grand public est, notamment, le disque dur.}
\entreeglossaire{Fork}{Mot anglais signifiant «~fourche~» ou «~fourchette~», un fork correspond à un nouveau projet créé à partir d'un autre à l'identique, sans détruire ce dernier. Cela implique que les droits accordés par les auteurs le permettent : ils doivent autoriser la modification, l'utilisation et la redistribution du code source. C'est pour cette raison que les embranchements se produisent facilement dans le domaine des logiciels libres.}
